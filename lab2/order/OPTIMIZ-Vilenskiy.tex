\documentclass[a4paper, 14pt]{extarticle}

% Поля
%--------------------------------------
\usepackage{geometry}
\geometry{a4paper,tmargin=2cm,bmargin=2cm,lmargin=3cm,rmargin=1cm}
%--------------------------------------


%Russian-specific packages
%--------------------------------------
\usepackage[T2A]{fontenc}
\usepackage[utf8]{inputenc} 
\usepackage[english, main=russian]{babel}
%--------------------------------------

\usepackage{textcomp}

% Красная строка
%--------------------------------------
\usepackage{indentfirst}               
%--------------------------------------             


%Graphics
%--------------------------------------
\usepackage{graphicx}
\graphicspath{ {./images/} }
\usepackage{wrapfig}
%--------------------------------------

% Полуторный интервал
%--------------------------------------
\linespread{1.3}                    
%--------------------------------------

%Выравнивание и переносы
%--------------------------------------
% Избавляемся от переполнений
\sloppy
% Запрещаем разрыв страницы после первой строки абзаца
\clubpenalty=10000
% Запрещаем разрыв страницы после последней строки абзаца
\widowpenalty=10000
%--------------------------------------

%Списки
\usepackage{enumitem}

%Подписи
\usepackage{caption} 

%Гиперссылки
\usepackage{hyperref}

\hypersetup {
	unicode=true
}

%Рисунки
%--------------------------------------
\DeclareCaptionLabelSeparator*{emdash}{~--- }
\captionsetup[figure]{labelsep=emdash,font=onehalfspacing,position=bottom}
%--------------------------------------

\usepackage{tempora}

%Листинги
%--------------------------------------
\usepackage{listings}
\lstset{
  basicstyle=\ttfamily\footnotesize, 
  %basicstyle=\footnotesize\AnkaCoder,        % the size of the fonts that are used for the code
  breakatwhitespace=false,         % sets if automatic breaks shoulbd only happen at whitespace
  breaklines=true,                 % sets automatic line breaking
  captionpos=t,                    % sets the caption-position to bottom
  inputencoding=utf8,
  frame=single,                    % adds a frame around the code
  keepspaces=true,                 % keeps spaces in text, useful for keeping indentation of code (possibly needs columns=flexible)
  keywordstyle=\bf,       % keyword style
  numbers=left,                    % where to put the line-numbers; possible values are (none, left, right)
  numbersep=5pt,                   % how far the line-numbers are from the code
  xleftmargin=25pt,
  xrightmargin=25pt,
  showspaces=false,                % show spaces everywhere adding particular underscores; it overrides 'showstringspaces'
  showstringspaces=false,          % underline spaces within strings only
  showtabs=false,                  % show tabs within strings adding particular underscores
  stepnumber=1,                    % the step between two line-numbers. If it's 1, each line will be numbered
  tabsize=2,                       % sets default tabsize to 8 spaces
  title=\lstname                   % show the filename of files included with \lstinputlisting; also try caption instead of title
}
%--------------------------------------

%%% Математические пакеты %%%
%--------------------------------------
\usepackage{amsthm,amsfonts,amsmath,amssymb,amscd}  % Математические дополнения от AMS
\usepackage{mathtools}                              % Добавляет окружение multlined
\usepackage[perpage]{footmisc}
%--------------------------------------

%--------------------------------------
%			НАЧАЛО ДОКУМЕНТА
%--------------------------------------

\begin{document}

%--------------------------------------
%			ТИТУЛЬНЫЙ ЛИСТ
%--------------------------------------
\begin{titlepage}
\thispagestyle{empty}
\newpage


%Шапка титульного листа
%--------------------------------------
\vspace*{-60pt}
\hspace{-65pt}
\begin{minipage}{0.3\textwidth}
\hspace*{-20pt}\centering
\includegraphics[width=\textwidth]{emblem}
\end{minipage}
\begin{minipage}{0.67\textwidth}\small \textbf{
\vspace*{-0.7ex}
\hspace*{-6pt}\centerline{Министерство науки и высшего образования Российской Федерации}
\vspace*{-0.7ex}
\centerline{Федеральное государственное бюджетное образовательное учреждение }
\vspace*{-0.7ex}
\centerline{высшего образования}
\vspace*{-0.7ex}
\centerline{<<Московский государственный технический университет}
\vspace*{-0.7ex}
\centerline{имени Н.Э. Баумана}
\vspace*{-0.7ex}
\centerline{(национальный исследовательский университет)>>}
\vspace*{-0.7ex}
\centerline{(МГТУ им. Н.Э. Баумана)}}
\end{minipage}
%--------------------------------------

%Полосы
%--------------------------------------
\vspace{-25pt}
\hspace{-35pt}\rule{\textwidth}{2.3pt}

\vspace*{-20.3pt}
\hspace{-35pt}\rule{\textwidth}{0.4pt}
%--------------------------------------

\vspace{1.5ex}
\hspace{-35pt} \noindent \small ФАКУЛЬТЕТ\hspace{80pt} <<Информатика и системы управления>>

\vspace*{-16pt}
\hspace{47pt}\rule{0.83\textwidth}{0.4pt}

\vspace{0.5ex}
\hspace{-35pt} \noindent \small КАФЕДРА\hspace{50pt} <<Теоретическая информатика и компьютерные технологии>>

\vspace*{-16pt}
\hspace{30pt}\rule{0.866\textwidth}{0.4pt}
  
\vspace{11em}

\begin{center}
\Large {\bf Лабораторная работа № 2} \\ 
\large {\bf по курсу <<Методы оптимизации>>} \\
\large <<Алгоритмы поиска экстремума унимодальной функции>> 
\end{center}\normalsize

\vspace{8em}


\begin{flushright}
  {Студент группы ИУ9-82Б Виленский С. Д. \hspace*{15pt}\\ 
  \vspace{2ex}
  Преподаватель Посевин Д. П.\hspace*{15pt}}
\end{flushright}

\bigskip

\vfill
 

\begin{center}
\textsl{Москва 2024}
\end{center}
\end{titlepage}
%--------------------------------------
%		КОНЕЦ ТИТУЛЬНОГО ЛИСТА
%--------------------------------------

\renewcommand{\ttdefault}{pcr}

\setlength{\tabcolsep}{3pt}
\newpage
\setcounter{page}{2}

\section{Задание}\label{Sect::task}

Определить, является ли функция унимодальной и выпуклой.
Сравнить скорость сходимости алгоритмов нахождения экстремума функции на отрезке: метод сходящихся отрезков, метод золотого сечения, метод разделением Фибоначчи.

\section{Результаты}\label{Sect::res}

Исходный код программы представлен в листингах~\ref{lst:code1}--~\ref{lst:code2}.

\begin{figure}[!htb]
\begin{lstlisting}[language={},caption={Нахождение минимумов функции},label={lst:code1}]
function derivativeAtPoint(f, x)
    delta = 1e-3
    return (f(x + delta) - f(x)) / delta
end

function secondDerivativeAtPoint(f, x)
    delta = 1e-3
    return (derivativeAtPoint(f, x + delta) - derivativeAtPoint(f, x)) / delta
end

function checkUnimodal(f, a, b, step, eps)
    extremum = nothing

    for x in a:step:b
        if abs(derivativeAtPoint(f, x)) < eps
            if !isnothing(extremum)
                return nothing
            end
            extremum = x
        end

        if secondDerivativeAtPoint(f, x) <= 0
            return nothing
        end
    end

    return extremum
end

function findExtrBySegments(f, a, b, step, eps)
    iters = 0

    while abs(a - b) > eps
        x1 = a + (b - a) / 3
        x2 = a + (b - a) * 2 / 3

        if f(x1) > f(x2)
            a = x1
        else
            b = x2
        end

        iters += 1
    end

    return (a + b) / 2, iters
end
\end{lstlisting}
\end{figure}
\begin{figure}[!htb]
\begin{lstlisting}[language={},caption={Нахождение минимумов функции},label={lst:code1}]
function findExtrByGoldRatio(f, a, b, step, eps)
    iters = 0

    goldRatio = (5^.5 - 1) / 2
    x1 = a + (1 - goldRatio) * (b - a)
    x2 = a + goldRatio * (b - a)
    x = (a + b) / 2

    while abs(a - b) > eps
        if f(x1) > f(x2)
            x = x2
            a = x1
            x1 = x2
            x2 = a + b - x2
        else
            x = x1
            b = x2
            x2 = x1
            x1 = a + b - x1
        end

        iters += 1
    end

    return x, iters
end

function findExtrByFibbonachi(f, a, b, step, eps)
    iters = 0

    fib1, fib2, fib3 = 0, 1, 1
    for i in 1:16
        fib1 = fib2
        fib2 = fib3
        fib3 = fib1 + fib2
    end
    x1 = a + (fib1 / fib3) * (b - a)
    x2 = a + b - x1
    x = (a + b) / 2

    while abs(a - b) > eps
        if f(x1) > f(x2)
            x = x2
            a = x1
            x1 = x2
            x2 = a + b - x2
        else
            x = x1
            b = x2
            x2 = x1
            x1 = a + b - x1
        end

        iters += 1
    end

    return x, iters
end

f = x -> ((x - 9.876)^2 + 12.345)
checkUnimodal(f, -100, 100, 2e-3, 1e-3)
println(findExtrBySegments(f, -10, 10, 2e-3, 1e-3))
println(findExtrByGoldRatio(f, -10, 10, 2e-3, 1e-3))
println(findExtrByFibbonachi(f, -10, 10, 2e-3, 1e-3))
\end{lstlisting}
\end{figure}

Результат запуска представлен в листинге ~\ref{fig:img1}.

\begin{figure}[!htb]
\begin{lstlisting}[language={},caption={Нахождение минимумов функции},label={lst:code1}]
9.876

(9.876023150046766, 25)
(9.875775199394639, 21)
(9.87616099071189, 18)
\end{lstlisting}
\end{figure}

\end{document}
